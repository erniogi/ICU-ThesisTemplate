\documentclass[12pt]{report}

\usepackage{titlesec}
% \usepackage[pdftex]{graphicx,xcolor}% for pdflatex
\usepackage[dvipdfmx]{graphicx,xcolor}% for platex or uplatex
\usepackage{setspace}
\usepackage[a4paper,top=25mm,bottom=25mm,left=30mm,right=25mm]{geometry}
\usepackage{here}
\usepackage{indentfirst}



\titleformat{\chapter}[hang] 
{\normalfont\huge\bfseries}{\thechapter. }{1em}{} 

\title{Motoori Norinaga’s Image of Man}
\author{Motoko Kokusai}
\date{}

\begin{document}

\input{titlepage.tex}
\doublespacing


\setcounter{tocdepth}{3}
\tableofcontents

\chapter{Introduction}
There are many ways to approach a study of Motoori Norinaga (1730-1801). Norinaga was
a prolific, learned, and original writer, and his ideas constitute important contributions to a
variety of themes in Japanese intellectual history. These include especially his work as a
kokugakusha, or scholar of national learning, his relation to and rejection of Confucianism,
his philological methodology, his nationalism, his development of the concept of the Shinto
“Ancient Way” (kodō), and his emphasis on certain aesthetic and humanistic values, notably
the concept of mono no aware, through a study of classical Japanese literature.

At a fundamental level these themes are necessarily intertwined in his work. For
example, his literary scholarship utilizes his philological method carefully, and his “Ancient
Way” is based on his scholarship concerning the Kojiki (Record of Ancient Matters, compiled
in 712), and moreover certainly also deeply concerns his perceptions of Confucianism and
Buddhism. As Muraoka Tsunetsugu notes, both philology and Kokugaku scholarship in
Norinaga's time had not yet been classified into specific academic disciplines; Norinaga
freely mixed these ideas and themes in his writings, and each served to shed illumination on
the others.1

Yet each theme in itself is worthy of careful study by contemporary scholars. The last of
these two themes, Norinaga's concepts of the “Ancient Way” and mono no aware, deserve
particular attention as they occupy places of increasing importance in Norinaga's scholarship
over the years. For this reason, it makes sense to view the rest of his ideas in relationship to
these two main concepts. In Norinaga's own writing, all of his other themes gradually take
a subordinate position and serve to support his clarification of mono no aware and the
“Ancient Way.”


\chapter{“Makoto”}
The overview of ICU logo can be seen in Figure~\ref{1.1}.

Cite test\cite{ishikawa1984}. 

\begin{figure}[H]
    \begin{center}
    \includegraphics[width=\textwidth]{../figures/ICU_japan.png}
    \end{center}
    \caption{ICU logo}
    \label{1.1}
\end{figure}

\chapter{“Fuga”}
XXXX
\section{Methods}


\chapter{“Mono no Aware”}
XXXX

\chapter{Norinaga’s Nationalism Reexamined}

\chapter{Conclusion}
\bibliographystyle{apalike}
\bibliography{ref}



\chapter*{和文抄訳}
% \addcontentsline{toc}{section}{和文抄訳}
本居宣長(1730-1801)のことを研究するにはいろいろな方法がある。
村岡典嗣が書いているように、宣長の時代には、国学と文献学がまだ特定的な学理的な分野として分けられていなかったため、
宣長にはいろいろなアイディアとテーマが自在に混じっている。
そのために、宣長の達成を考えてみるとき、
儒学の拒絶、文献学の方法、国学主義における「国体」という考え方の原理的な形成、
神道における「古道」の概念の展開、
それに古典の研究によって、特に「物のあわれ」などという宣長が考え出した人道主義と文芸的な価値、
など様々な見地から宣長を研究することができるだろう。

宣長の経歴を考えてみると、これほど幅広いテーマがあること自体は驚くべきことではない。
1730 年に松坂に生まれた宣長は、若いときから、仏教と神道を学んだ。
両親は浄土宗の信者であり、そして、松坂の近くには天照大神崇拝の一番大切な場所である伊勢神宮がある。
1752 年に医学を学ぶため宣長は京都に行った。
京都では、医学以外の勉強もたくさんした。
特に、堀景山に師事して、契沖(1640-1701)と荻生徂徠(1666-1728)の書を学んだ。

僧侶で万葉集の注釈を書いた契沖は、明らかに宣長の文献学の方法に、大変な影響を与えた。
もう一人の徂徠という儒学者は、朱子学を批判し、儒教の古典に戻って本当に日本的な儒学を作ろうとした「古学」という学問の提唱者であった。
宣長は徂徠を強く批判したものの、彼は徂徠朱子学を拒絶するという方法に影響を受けた、ということを丸山真男が詳しく描いている。
宣長に影響を与えた人物として忘れてはならない人物が、もう一人が残っている。
それは国学の加茂真淵(1697-1769)である。宣長の「古道」は真淵に起源があると思われるのである。
宣長と真淵は 1763 年に松坂で会った。彼らは互いの生涯で、その一度しか会わなかったが、
宣長はそのあと真淵の弟子だと称して、よくお互いに手紙を送り合い、真淵は宣長の古事記の注釈を励ました。

このように、仏教、神道、儒学が宣長の身近にあった。
そして、契沖、徂徠、真淵の影響を背景として、宣長は彼自身の研究を始めた。
宣長の著作を見てゆくと、最初のうちから彼特有の根本的なアイディアは揃っており、それがだんだんと固まっていき、
\end{document}